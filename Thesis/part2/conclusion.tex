\chapter*{Conclusion of Part~\ref{part:data_integration}}

In this part, we presented the various parts required to enable Data Integration in the enterprise.

First, we created an internal knowledge base by importing DBpedia into SAP HANA. On top of that, we built a set of services that enable entity disambiguation, semantic enrichment and schema matching. We presented RUBIX, a framework enabling mashup of potentially noisy enterprise and external data. The implementation is based on Open Refine and uses our entity disambiguation service to annotate data with rich types. As a result, the matching process of heterogeneous data sources is improved. We have also shown that it is possible to reveal what are the ``important'' properties of entities by reverse engineering the choices made by Google when creating knowledge graph panels and by comparing  users preferences obtained from a user survey. Our motivation is to represent this choice explicitly, using the Fresnel vocabulary, so that any application could read this configuration file for deciding which properties of an entity is worth to visualize.

Last, we cover the aspect of integrating external data coming from social media outlets. Data nowadays is spread over heterogeneous silos of archived and live data. People willingly share data on social media by posting news, views, presentations, pictures and videos. We presented SNARC, a service that uses semantic web technology and combines services available on the web to aggregate social news. SNARC brings live and archived information to the user that is directly related to his active page.

Going back to our scenario, the proposed frameworks and services will allow our analyst \textbf{Dan} to be able to find and match various reports he is working on. Moreover, he will be able to augment extra measures and dimensions to his reports using DBpedia. In addition, \textbf{Dan} will be able to monitor relevant social feeds. This will allow him to either embed social snippets directly in his reports or to discover new information sources.